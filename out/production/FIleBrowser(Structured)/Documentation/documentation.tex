\documentclass[10pt]{article}

\usepackage{array}
\usepackage{color}

\newcommand{\blueIt}[1]{\itshape \textcolor{blue}{#1}}

\begin{document}


\title{\bfseries \Huge Documentation and Design Pattern Usage on\\
\blueIt{``File Explorer"}}

\author{\bfseries Muhtasim Ulfat Tanmoy\\
1405086}
\date{\today}

\maketitle


\newpage

\section {Design Pattern Identification and Selection}

The given file Explorer project has following requirements:

\begin{itemize} %unordered list
 \item {\bfseries TreeView } - for showing file hierarchy.
 \item {\bfseries TabletView } - for showing file in a table.
  \item {\bfseries TileView } - for showing file in a tiles.
  \item {\bfseries Switching } - convenient way for transition between views.
 \end{itemize}
 
 Now, {\bfseries TreeView } uses the composite design pattern.As we can create different node and each node can also have other nodes as child it matches all the specification for {\bfseries Composite design pattern}.\\
 
 {\bfseries TabletView } and {\bfseries TileView }  has list of files that has been adapted to the view in our code through fileModel class(works as adaptee ) ,this clearly represents the {\bfseries Adapter class pattern}.\\
 
 
 
 {\bfseries Switching } between views like making the same set of of files appear in different form in front of users clearly references {\bfseries Factory design pattern}.
 
 
 So,TreeView class in javafx implements the composite pattern,the Tableview class does the adapter pattern and differnt view switching implements the factory design pattern.The main function that is initialized only once during the lifetime of the app implements singleton degign pattern.\\
 
 \newpage
 
 





\section {Design Pattern Usage	}


The design pattern usage on our project is described with relevent classes below:\\


{\bfseries The Main class} runs only once and it sets the fxml in the app.It sets the resolution and therefore implements Singleton Pattern.\\

{\bfseries The Controller class} controls the whole app with different properties.\\


{\bfseries The Treeview} API is used here to show the hierarchy of folders to navigate to.This treeview implements composite design pattern.\\



The {\bfseries myTreeItem} class implements all the functionalities of a single tree node or element.It's onClick EventHandler implements what to do when an item is clicked.The folder icon is changed when a tree node having children is expanded or closed.And also a list of all available childen including folder and file is shown\\

Now there are two ways of showing the all the file list in the current working directory.

{\bfseries Tableview :} Here all the files are are adapted to Tableview through fileModel class and CellValueFactory.

The {\bfseries FileModel class} extends the File class API and relevent file values are filtered througnh this class to our Tableview class.\\

{\bfseries Tableview :} Here all the files are adapted to Tilepane view through TileItems.Here each tileItems are actually a vBox API class.

{\bfseries TileItem :} class extends vBox and defines an structure for each item in our tileView.It autometically sets image for file or directory.It also implements {\bfseries onClickListener} for easy transition to different directories.\\

{\bfseries The above two methods implement Adapter class}


{\bfseries Switching View :} Done through ViewFactoryPattern Class.This class takes the view and sets them accordingly.ToggleGroup[ has been used for switching views.


\newpage





\section {Design Pattern Implementation}

The design pattern implementation is provided in code.

Back functionality is used to to upper hiererchy.

User can also go to differnt folders by typing the path to  that folder.

And a meaningful function name has been used and proper documentation is given as comment beside every function and variable declaration.






	
\end{document}